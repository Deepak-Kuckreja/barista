\documentclass[a4paper]{article}
\usepackage[utf8]{inputenc}
\usepackage[margin=1in]{geometry}

\title{\textsc{Barista: A Distributed, Synchronously Replicated, Fault Tolerant, Relational Data Store}}
\author{Anant Bhardwaj \\ anantb@csail.mit.edu \\ MIT CSAIL \and Rebecca Yale Taft \\ rytaft@mit.edu \\ MIT CSAIL \and Manasi Vartak \\ mvartak@mit.edu \\ MIT CSAIL}
\date{Spring 2014}

\begin{document}
\maketitle

\begin{abstract}
Barista is a distributed, synchronously replicated, fault tolerant, relational data store. It is a layer written over postgres that manages realtime replication of data in a distributed infrastructure to provide fault-tolerance and load balancing. All writes are propagated synchronously using paxos. Barista exposes SQL for data management. Client applications can use the same SQL with Barista and under the hood it takes care of load balancing, consistency, and fault-tolerance seamlessly.
\end{abstract}
\section{Introduction}




\end{document}